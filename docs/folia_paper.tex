\documentclass[a4paper,10pt,twoside]{article}
\usepackage[english]{babel}
\usepackage{hyperref}{}
\usepackage{clin}
\usepackage{harvard}
\usepackage{listings}
\usepackage{color}
\lstset{% general command to set parameter(s)
basicstyle=\footnotesize,
keywordstyle=\color{black}\bfseries\underbar,
identifierstyle=\color{black}\bfseries\underbar,
stringstyle=\ttfamily,
inputencoding=utf8,
extendedchars=true,
literate={á}{{\'a}}1 {ã}{{\~a}}1 {é}{{\'e}}1 {ˈ}{{\textipa{'}}}1 {ʊ}{{\textipa{U}}}1 {ː}{{\textipa{:}}}1 {ɝ}{{\textipa{O}}}1 {æ}{{\ae}}1,
}


\pagestyle{empty}

\begin{document}


\title{FoLiA: A practical XML Format for Linguistic Annotation - a descriptive and comparative study}

\author{Maarten van Gompel \email{proycon@anaproy.nl}\\
{\normalsize \bf Martin Reynaert} \email{reynaert@uvt.nl}
\AND \addr{Centre for Language Studies\\ 
Radboud University Nijmegen} \AND \addr{Tilburg Center for Cognition and 
Communication\\Tilburg University} }



\maketitle
%\thispagestyle{empty} 

\begin{abstract}

In this paper we present FoLiA, a Format for Linguistic Annotation, and conduct
a comparative study with other annotation schemes, including the Linguistic
Annotation Framework (LAF), the Text Encoding Initiative (TEI) and Text Corpus
Format (TCF). An additional point of focus is the interoperability between
FoLiA and metadata standards such as the Component MetaData Infrastructure
(CMDI), as well as data category registries such as ISOcat. The aim of the
paper is to present a clear image of the capabilities of FoLiA and how it
relates to other formats. This should open discussion and aid users in their
decision for a particular format.

FoLiA is a practically-oriented XML-based annotation format for the
representation of language resources, explicitly supporting a wide variety of annotation
types. It introduces a flexible and uniform paradigm and a representation
independent of language or label set. It is designed to be highly expressive,
generic, and formalised, whilst at the same time focussing on being as
practical as possible to ease its adoption and implementation. The aspiration
is to offer a generic format for storage, exchange, and machine-processing of
linguistically annotated documents, preventing users as well as software tools
from having to cope with a wide variety of different formats, which in the
field regularly causes convertibility issues and proliferation of ad-hoc
formats. FoLiA emerged from such a practical need in the context of
Computational Linguistics in the Netherlands and Flanders. It has been
successfully adopted by numerous projects within this community. FoLiA was
developed in a bottom-up fashion, with special emphasis on software libraries
and tools to handle it.

\end{abstract}

\section{Introduction}

Linguistic annotation is the enrichment of data with features that describe or
analyse this data linguistically. The data often takes the form of written
text, which may well be a transcription of an audio/video clip of human speech.
A wide variety of such linguistic features can be thought of, examples are
part-of-speech annotation, lemmatisation, co-reference relations, semantic
roles, and so on. Within the field of computational linguistics we often see
tools and data collections such as corpora use ad-hoc formats that are limited
to a mere subset of these \emph{annotation types}. When newer annotation types
become available at a later point in time, the original format needs to be
extended, which often occurs in a similar ad-hoc fashion and raises concerns of
backward-compatibility. Moreover, such ad-hoc formats are often not formalised
in any schema definition and thus hard to validate. This is all very well if
this usage is limited to temporary data formats in a linguistic processing
pipeline with specialised tools, but the practice becomes problematic when
these formats are used as a data exchange or corpus storage format. This
becomes even more of an issue if the ad-hoc formats are the result of the
processing of a richer input format but fail to retain parts of the input
format or the means to effectively refer to such original input format.  Data
loss is one of the most undesirable outcomes. A data exchange or corpus storage
format requires a more extensible and a more formalised format, one that does
not reach its limits as different project partners or projects include
different annotation types, and that never loses data with respect to the
original input source.

In this paper we propose and describe FoLiA: Format for Linguistic Annotation,
as such a format for data exchange and storage, as well as for processing by
tools. The prime motivation for the development of FoLiA stems from the
grievances we just sketched, and which we found all too common in practice.
Nevertheless, ad-hoc formats are often used because they are simple and
practical. If we intend to offer a viable alternative to this we must ensure to
strongly focus on practical needs and ease of adoption and usage, as well as
make available the necessary tools to work with the format. 

We observe and acknowledge that numerous annotation formats exist in the field
of Computational Linguistics and Linguistics in general. This raises the
question why we have opted to create a new format rather than to use an
existing one. FoLiA emerged in the context of the computational linguistics
community in the Netherlands and Flanders. It was presented at the
Computational Linguistics in the Netherlands (CLIN) conference in 2011 and
2012. A prevalence of the above mentioned ad-hoc formats was observed in
many projects and tools, creating problems for data exchange and corpus storage.
Development of FoLiA was initiated at an intersection of several projects
(Stevin project SoNaR \cite{StevinSONAR2013}, CLARIN-NL projects TICCLops and
TTNWW, NWO project DutchSemCor)\footnote{We refer to Section~\ref{sec:practice}
for a discussion of FoLiA in practice} and software projects (Ucto, Frog,
Valkuil) that all faced such issues. At that point in time, the largest
available Dutch language corpus, the D-Coi corpus, was using a custom XML-based
format \cite{DCOI} that could be considered a de facto standard.
However, this format was severely limited in expressivity.  FoLiA was built on
the foundations laid by the D-Coi corpus format, extending it way beyond the
limits this format imposed. Another widely established format in the same
community is the Alpino XML format, used for the encoding of (Dutch) syntactic
structures and dependency parses by the widely used Alpino parser
\cite{ALPINO}. This format is good at representing what it was designed for,
but not expressive enough to act as a more generic format for linguistic
annotation.

Existing formats that may compete with FoLiA feature-wise seem not to have been
represented well in the aforementioned community, in part maybe simply due to a
lack of awareness and lack of interest in formatting issues. Another reason,
from a more development-oriented perspective, is the lack of NLP-tools and
software libraries that offer direct support for existing formats and thus make
it easy to work with them. FoLiA was consciously designed with this in mind, in
a very bottom-up fashion, right alongside tools and libraries and within the
scope of actual NLP projects. We see this as an important prerequisite to
facilitate adoption and, by extension, de facto standardisation. This approach
seems successful and noted. \citeasnoun{CLARINDEV12} writes in a paper on
recent developments in CLARIN-NL: ``the corpus format in use in this project
[the TTNWW project], FoLiA, appears to develop into a de facto standard for
text corpora annotation for Dutch''. We dedicate Section~\ref{sec:practice} to
a more detailed discussion of the usage of FoLiA in current projects and
lessons learned.

In Section~\ref{sec:comparison} the differences and compatibility between FoLiA
and the main alternatives available will be studied in-depth. For now, before
going into an actual description of the FoLiA paradigm in
Section~\ref{sec:description}, we will list
several dimensions or criteria on which formats for linguistic annotation can
be assessed in Section~\ref{sec:criteria}.

\section{Criteria for Assessment}
\label{sec:criteria}

We draw the dimensions on which we assess formats for linguistic annotation
from \citeasnoun{LAF}. In this study a group of experts drafted requirements
for a framework for linguistic annotation. We interpret these as dimensions or
criteria along which the characteristics of annotation formats can be defined.
Each dimension we paraphrase as a question. Subsequently, we give an indication
of how FoLiA measures along each of these dimensions. This provides an initial
overview of FoLiA and presents a foundation for further comparison in
Section~\ref{sec:comparison}.

\begin{enumerate}
\item \textbf{Uniformity \& Consistency} -- To what extent is the format
  uniform; based upon the same building blocks rather than ad-hoc provisions?
  Is the format capable of expressing different levels of granularity? We would
  like to group this category with what \citeasnoun{LAF} denote as
  \emph{consistency}: Are the same mechanisms used to indicate the same type of
  information?
  \begin{itemize}
    \item[] FoLiA proposes a uniform paradigm to represent linguistic data, this
      paradigm or data model consists of a finite set of building blocks that
      are used throughout the format. Section~\ref{sec:description} will focus
      on a detailed description of this model.
  \end{itemize}
\item \textbf{Expressivity} -- Is the format expressive enough to represent a
  wide variety of linguistic annotation?
  \begin{itemize}
    \item[] Linguistic annotation types need to be explicitly supported in FoLiA,
      and a wide variety indeed is, following a uniform paradigm. FoLiA is a
      verbose format but tries to not force undesired verbosity on the users
      if the user is willing to compromise other dimensions such as formality.
  \end{itemize}
\item \textbf{Openness} -- Does the representation commit to a single linguistic
  theory, or is it open to multiple theories and label sets? Likewise, we add,
  is the format specialised for one or more languages or is it language
  independent?
  \begin{itemize}
    \item[] FoLiA does not commit to any single linguistic theory and is open to
      any label set or language. The definition of label sets is separate from
      the annotated documents. Section~\ref{sec:setdefinitions} will discuss
      this.
  \end{itemize}
\item \textbf{Extensibility} -- Is the format easily extensible?
  \begin{itemize}
    \item[] The openness dimension sketched above already allows for a fair
      amount of flexibility. In addition to this, higher-order annotations (see
      Section~\ref{sec:features}) in FoLiA allow for additional freedom by
      being able to freely associate extra features (key/value pairs) with any
      annotation, without concession to the formalised nature of the format.
      The main annotation types, however, are a core specialised part of the format and are
      defined centrally. This distinguishes FoLiA from more abstract formats.
      FoLiA remains in constant development and subsequent releases often add
      support for additional annotation types, fitted within the FoLiA paradigm
      and with full regard for maintaining backward-compatibility. Users who
      see the need for additional annotation types are always encouraged to
      contact the authors.
  \end{itemize}
\item \textbf{Explicitness \& Formality} -- Is the annotation scheme explicit
  and unambiguous enough, not shifting the burden of interpretation to
  processing software? To this we add what \citeasnoun{LAF} denote as
  \emph{Semantic Adequacy}: do the representational structures and data
  categories have a proper formal semantics? Are they stored in a shared
  fashion? In this same spirit we further add: is there a proper schema definition of the
  format against which instances can be validated?
  \begin{itemize}
    \item[] FoLiA is a highly verbose format and explicit in its representations.
      There is a strict separation between the annotated document and the
      definition of the data categories in a separate FoLiA Set Definition
      format. This allows for high formality and deep validation. At the same
      time, this is not forced on users so more informal usage without set
      definitions is also possible. Shallow validation validates any FoLiA
      document against a RelaxNG schema \cite{RELAXNG}. Each FoLiA element has clearly
      defined semantics, defined in the documentation \cite{FOLIA}.
  \end{itemize}
\item \textbf{Human readability} -- To what extent are the representations human readable and editable?
\begin{itemize}
    \item[] FoLiA XML files are human readable, though high verbosity may
      complicate matters. On the other hand, we strongly commit to usage of XML hierarchical notation
      and inline annotation (where possible), which makes it relatively easier
      compared to certain stand-off-only formats. 
  \end{itemize}
\item \textbf{Media independence} -- Does the annotation format support
  different media? Or is it committed to one, such as written text?
  \begin{itemize}
    \item[] Though different media are supported and proposals are being written
      for more comprehensive speech annotation, written text is currently best
      represented. 
   \end{itemize}
\item \textbf{Incrementality} -- Does the format support annotations of
  alternatives and partial/under-specified results? 
   \begin{itemize} 
     \item[] FoLiA supports the annotation of alternatives for all
    main annotation types. These alternatives can each be assigned confidence
    scores. FoLiA also supports explicit corrections, or mere suggestions for
    corrections, on annotations.
    Untokenised text, and text at different stages of processing can also be
    encoded. There is an important limitation in this dimension, however. FoLiA
    presupposes one authoritative tokenisation if linguistic annotations are
    added. Increased granularity within this tokenisation, through morphemes or
    phonemes, is always possible.
  \end{itemize}
   \newcounter{enumTemp}
   \setcounter{enumTemp}{\theenumi}
\end{enumerate}

We add several criteria of our own to the requirements carefully drafted by
\citeasnoun{LAF}. These were not addressed as such in their study, but are
considered important nevertheless:

\begin{enumerate}
\setcounter{enumi}{\theenumTemp}
\item \textbf{Genericness vs Specialisation} -- How generic is the format? Is
  it perhaps highly specialised for a certain kind of annotation? This is
  strongly correlated with
  \emph{expressivity}.
  \begin{itemize}
    \item[] Our aim is to provide, within reasonable limits, a
      one-format-fits-all solution. The scope of FoLiA is broad rather than
      narrow and highly specialised.
   \end{itemize}
\item \textbf{Parsability \& Efficiency} -- Can the format be parsed and
  processed efficiently? Does it make a big demand on system resources?
  \begin{itemize}
    \item[] We emphasise expressivity over efficiency. That is, efficiency may
      never come at the cost of expressivity. Nevertheless, we still
      distinguish ourselves from several other formats by the way the data is
      structured as a mix of inline and offset annotation. This is done to
      facilitate efficiency/parsability and also has a positive effect on human
      readability. FoLiA is designed also with streaming XML parsers in mind,
      which do not hold the entire document in memory. Keeping entire XML
      documents in memory, FoLiA documents included, is often very expensive.
  \end{itemize}
\item \textbf{Locality} - Are annotations pertaining to the same sequence(s) in
  close proximity to one another in the representation, or are they spread out
  over the full representation? And is the representation spread over multiple
  files?
  \begin{itemize}
    \item[] Annotations can be grouped in various ways. In FoLiA we choose to
      keep information as local as possible, and within a single XML file.
  \end{itemize}
\item \textbf{Accessibility} - Is the proposed format available freely (open
  access), both free of charge as well as free to extend, and are there tools and
  software libraries available for working with the format, released under
  open-source licenses? 
  \begin{itemize}
    \item[] This for us is an important dimension that is highly emphasised. Our
      aim is to deliver not just a theoretical framework, but a practical
      format with tools and libraries under open-source licenses.
  \end{itemize}
\item \textbf{Abstraction} - How far does the format abstract from the
  linguistic concepts familiar in the field?
  \begin{itemize}
    \item[] Certain formats require high levels of abstraction whilst others are
      very close to the original source. The usage of XML already implies a higher level of abstraction. We keep
      the XML elements strongly correlated with meaningful concepts in the
      field. Annotation types and text-structural elements are explicitly
      defined and fixed by the format, rather than completely open to the user
      to define. In this regard, we choose a lower level of abstraction than
      certain other formats. 
  \end{itemize}
\end{enumerate}

Formats in the field will rank differently along each of these dimensions. A
precise objective quantification along these dimensions is neither possible nor
necessary to illustrate and compare the properties of a format. Moreover,
different objectives result in a different prioritisation of the dimensions.
This will, in part, account for the variation in formats in existence.


\section{Description}
\label{sec:description}

An extensive overview of all annotation types FoLiA offers is beyond the scope
of this paper, for that we refer to the FoLiA Documentation \cite{FOLIA} and
the validation schema which is available in RelaxNG format \cite{RELAXNG}, the latter is
ideally suited to validate your own FoLiA documents to make sure they adhere to
the standard. In the current section our aim is to clarify the data model used by
FoLiA, show the capabilities given this paradigm, and motivate our choices.

\subsection{Annotation style}

FoLiA is an XML-based format \cite{XML}. XML is a common choice as it is a
widely established and accepted standard. Choosing XML also implies that we
have powerful technologies such as XPath, XSLT, and XQuery at our disposal for
handling FoLiA documents. XML is inherently a hierarchical format which lends
itself well for the encoding of text documents. This is proven most of all by
the ubiquity and success of (x)HTML on the web.  The OHCO thesis, first posited
by \citeasnoun{OHCO1} and revised in \citeasnoun{OHCO2}, asserts that ``text is an Ordered
Hierarchy of Content Objects``. Each structural element that makes up the text
may hold deeper, more fine-grained, structures of the text. XML is well suited
for modelling such data and FoLiA makes strong use of the hierarchical
capabilities of XML. For document structure it fully commits to the OHCO
thesis, and inherits most of this from its predecessor D-Coi. As far as
possible FoLiA continues in this fashion for linguistic annotation, which
implies an \emph{``inline annotation''} style. We can distinguish four categories of
annotation or representation in FoLiA, the first two of which are forms of
inline annotation:

\begin{enumerate}
\item \textbf{Structure annotation} -- Representation of document structure.
  This includes \emph{words/tokens, divisions (i.e. chapters, sections,
  subsections), headers (titles of divisions), paragraphs, sentences,
  utterances, morphemes,
  phonemes, lists,
  list items, figures and their captions, extra linebreaks, gaps and events.}
\item \textbf{Token annotation} -- Annotation applied to a single token or, by
  extension, to another structural element. This category encompasses the
  following annotation types: \emph{part-of-speech tags, lemmatisation, lexical
  semantic sense annotation, subjectivity annotation, domain annotation}. All
  are annotations pertaining to a single structural element.
  \newcounter{enumTemptwo}
  \setcounter{enumTemptwo}{\theenumi}
\end{enumerate}

The choice for inline annotation for structure annotation is largely justified
by the OHCO thesis and the aforementioned success of hierarchical representations of text on
the web. For token annotation the choice for inline annotation maximises the
use of the hierarchical expressivity of XML, which results in better \emph{human
readability} and \emph{parsability}. The excerpt of FoLiA XML in
Figure~\ref{fig:tokenannotation} illustrates a sentence, tokenised, annotated with
Part-of-Speech tags and showing structure annotation as well as token annotation.

\begin{figure}[tbh]
\begin{lstlisting}[language=xml]
<s xml:id="example.p.1.s.2">
  <w xml:id="example.p.1.s.2.w.1">
    <t>He</t>
    <pos class="PRP" />
  </w>
  <w xml:id="example.p.1.s.2.w.2">
    <t>is</t>
    <pos class="VBZ" />
  </w>
  <w xml:id="example.p.1.s.2.w.3">
    <t>Mr.</t>
    <pos class="NNP" />
  </w>
  <w xml:id="example.p.1.s.2.w.4">
    <t>Smith</t>
    <pos class="NNP" />
  </w>
  <w xml:id="example.p.1.s.2.w.5">
    <t>.</t>
    <pos class="." />
  </w>
</s>
\end{lstlisting}
\caption{An excerpt of FoLiA XML showing a sentence, tokenised, with
part-of-speech annotation. The excerpt illustrates the use of in-line annotation for
structural elements and token annotation, and the use of identifiers.\label{fig:tokenannotation}}
\end{figure}

FoLiA elements may carry a unique identifier by which they can be easily
referenced, both internally and from external sources.
Figure~\ref{fig:tokenannotation} illustrates this. The precise convention
adopted for naming identifiers is left to the user.

The inline annotation style used in FoLiA's token annotation brings benefits
but is not without its limits. In order to attain the high expressivity we
desire, stand-off annotation can not be avoided. In the title of their study,
\citeasnoun{OHCO2} rightly posit that problems arise as soon as overlapping
structures have to be represented. FoLiA's \emph{Span Annotation} remedies
this problem by resorting to a stand-off annotation style in separate
annotation layers. Span annotation is used for annotation types that span
multiple structural elements.

\begin{enumerate}
\setcounter{enumi}{\theenumTemptwo}
\item \textbf{Span annotation} -- 
Annotation applied to a span of multiple tokens or other structural elements:
\emph{syntactic structure, named entities, dependency relations, co-reference
relations, semantic roles, chunking/shallow parses}.
\item \textbf{Higher-order annotation} -- Annotations on annotations. See Section~\ref{sec:higherorder}.
\end{enumerate}


Within the scope of a span annotation element, references are made to the
structural elements over which the span applies, often word tokens,
and refer to their unique identifiers. Span annotation elements of a certain
type are grouped in an \emph{annotation layer} corresponding to the annotation
type, this layer is in turn embedded \emph{inline} in the structural element
that encompasses all spans. An example of this is shown in
Figure~\ref{fig:spanannotation}. Span annotation is also geared towards nested,
tree-like structures, such as syntax.

\begin{figure}[tbh]
\begin{lstlisting}[language=xml]
<s xml:id="example.p.1.s.2">
  <w xml:id="example.p.1.s.2.w.1">
    <t>He</t>
    <pos class="PRP" />
  </w>
  <w xml:id="example.p.1.s.2.w.2">
    <t>is</t>
    <pos class="VBZ" />
  </w>
  <w xml:id="example.p.1.s.2.w.3">
    <t>Mr.</t>
    <pos class="NNP" />
  </w>
  <w xml:id="example.p.1.s.2.w.4">
    <t>Smith</t>
    <pos class="NNP" />
  </w>
  <w xml:id="example.p.1.s.2.w.5">
    <t>.</t>
    <pos class="." />
  </w>
  <entities>
        <entity xml:id="entity.1" class="person">
            <wref id="example.p.1.s.2.w.3" />
            <wref id="example.p.1.s.2.w.4" />
        </entity>
  </entities>
</s>
\end{lstlisting}
\caption{An excerpt of FoLiA XML illustrating the use of span annotation, in this example: named entities.\label{fig:spanannotation}}
\end{figure}

A deliberate choice was made to keep information together and as localised as
possible. The annotated document encompasses a single XML file rather than
multiple files for different annotation types. The annotation layers,  using
stand-off annotation, are embedded in an in-line fashion at a local level, as
we have seen. This choice is motivated largely by parsability and to a lesser
extent by human readability. Keeping information together and localised allows
parsers to function in a streaming fashion and lowers memory requirements. When
a parser retrieves for instance a sentence, it can be sure to have contained
within it all relevant annotations that are limited to the scope of that
sentence. Moreover, holding everything in a single file ensures the integrity
of the document and its annotations is maintained as a whole and can be easily
used in data exchange or storage, also with regard for versioning control. This
focus on high locality distinguishes FoLiA from various other formats, as shall
be discussed later in Section~\ref{sec:comparison}. 

\subsection{Text content} 

The textual content of a structural element is considered an annotation just as
well, albeit with certain special properties. The \texttt{t} element represents
the text content in FoLiA and is shown in Figures  \ref{fig:tokenannotation}
and \ref{fig:spanannotation}. Text content can be embedded on various levels of
structure annotation. Any text content outside of the word/token level is by
definition untokenised. Likewise, text content within the word/token level by
definition is tokenised. Links between the two can be retained by using the
optional \emph{offset} attribute, pointing to the unicode \emph{character} (as opposed
to byte offset) where the text fragment is embedded in the higher level. 
These character offset references \emph{by default} point to the text pertaining to
the structure element ``one level up'', where a level also corresponds to a
structure annotation element. Figure~\ref{fig:textcontent} illustrates this.

The same figure also illustrates that a token is not the structure element of
smallest granularity, there is also notation for representing morphemes and
phonemes. These are embedded within an annotation layer in a word/token and
can make use of the same offset attribute to relate their text to the full token.

\begin{figure}[tbh]
\begin{lstlisting}[language=xml]
<p xml:id="example.p.1">
 <t>He hits Mr. Smith. That came quite expected!</t>
 <s xml:id="example.p.1.s.1">
  <t offset="0">He hits Mr. Smith.</t>
  <w xml:id="example.p.1.s.1.w.1"><t offset="0">He</t></w>
  <w xml:id="example.p.1.s.1.w.2"><t offset="3">hits</t>
    <morphology> 
        <morpheme class="lexical" function="lexical">
         <t offset="0">hit</t>
        </morpheme>
        <morpheme class="suffix" function="inflectional">
         <t offset="3">s</t>
        </morpheme>
    </morphology>
   </w>
  <w xml:id="example.p.1.s.1.w.3"><t offset="8">Mr.</t></w>
  <w xml:id="example.p.1.s.1.w.4" space="no"><t offset="10">Smith</t></w>
  <w xml:id="example.p.1.s.1.w.5"><t offset="15">.</t></w>
 </s>
 ..
</p>
\end{lstlisting}
\caption{Text content with references on multiple levels: paragraph, sentence
and word. Also included is an example of morphological annotation.\label{fig:textcontent}}
\end{figure}


\subsection{Sets, classes and common attributes}
\label{sec:setdefinitions} 

Central to the FoLiA paradigm is the notion of \emph{sets} and \emph{classes}.
A set represents a label set for a particular annotation type, the elements of
this set are referred to as classes and represent the individual data
categories. FoLiA does not predefine any sets or classes, these are left to the
user to define, and in doing so we commit to a high level of openness. Set
definitions are defined outside of the annotated FoLiA documents, using a
different format: \emph{FoLiA Set Definition format}\footnote{Alternative set
definition formats may be supported as well in later versions of FoLiA}.

It is the task of the FoLiA Set Definition format to define all possible
classes in a set, or to declare a set as ``open'' if the classes do not form a
closed vocabulary. Whereas the classes are short identifiers, the set
definition format can map each of these to fully human-readable labels. A
mapping may also be included to an external data category registry such as
ISOcat \cite{ISOCAT} which may take over a lot of the role of the set
definition format.

The set definition format may also pose constraints on how classes of
\emph{subsets} may be combined. The notion of subsets is explored in 
section~\ref{sec:features}. The validity of a FoLiA document can be validated
against a RelaxNG schema which checks whether the document correctly adheres to
the proper FoLiA syntax, this we call ``shallow validation''. Set definition
files allow us to take a step further and do ``deep validation'' as well, in
which the actual classes are checked.

Figures
\ref{fig:tokenannotation} and \ref{fig:spanannotation} illustrate the use of
classes in FoLiA. The actual Part-of-Speech tags are the classes and they form part of a
set (in this case the Penn Treebank Tagset). In these excerpts, the sets are
not made explicit. Sets can be referred to using the \texttt{set} XML attribute
that any token annotation element or span annotation element may take. The
value of this attribute is a URL where the set definition resides. Each
annotation type with the used set must be \emph{declared} in the metadata portion of
the FoLiA document. If only one set is declared for a given annotation type, as
is often the case, then the \texttt{set} attribute needs not be specified on
each of the relevant annotation elements. These mandatory declarations will be
further discussed in Section~\ref{sec:metadata}. 

This paradigm is applied consistently to annotation elements of all types. The
set points to a URL holding a file in FoLiA Set Definition format that
declares all possible classes of the set. It may also declare a set as being
open. Human readable labels can be associated with each class in a set and
constraints can be posed on combinations of classes in subsets, see
Section~\ref{sec:features}. Moreover, references to a data category registry
such as ISOcat \cite{ISOCAT} can be made at this level as well.

The FoLiA paradigm furthermore states that annotation elements may always take certain
common XML attributes, these are never mandatory:

\begin{itemize}
\item \textbf{annotator} -- A name or ID of the person or system responsible for this annotation, i.e. the creator or editor.
\item \textbf{annotatortype} -- ``auto`` or ``manual'', indicating whether the annotator is an automated system or whether the annotation is the result of manual human labour.
\item \textbf{confidence} -- A value between 0 and 1 expressing the confidence placed in this annotation by the annotator.
\item \textbf{n} -- A string representing a numeral that indicates the position
  of this element in a certain order. Used in, for example, chapter numbers, section numbers, enumerations.
\end{itemize}

For structural elements in a speech context, additional common attributes are
available:

\begin{itemize}
  \item \textbf{src} -- A link to a sound or video clip containing the speech.
  \item \textbf{begintime} --- A timestamp indicating the begin time of this speech act.
  \item \textbf{endtime} --- A timestamp indicating the end time of this speech act.
\end{itemize}

\begin{figure}[tbh]
\begin{lstlisting}[language=xml]
<w xml:id="example.p.2.s.1.w.2">
  <t>amazing</t>
  <pos set="http://someserver/claws" 
    class="jj" annotator="Maarten van Gompel"
    annotatortype="manual" confidence="0.9"/>
  <lemma set="http://someserver/lemmas"
    class="amazing" annotator="C3PO" 
    annotatortype="auto" />
  <sense set="http://someserver/wordnet3"
    class="amazing%3:00:00:surprising:00"/>
  <alt auth="no">
   <sense set="http://someserver/wordnet3"
     class="amazing%3:00:00:impressive:00"
     annotator="Martin Reynaert" annotatortype="manual"
     confidence="0.23" />
  </alt>
</w> 
\end{lstlisting}
\caption{Excerpt of a word with token annotations in (fictitious) sets, with common attributes. Also shown is an example of an alternative.\label{fig:wordannotation}} 
\end{figure}


Figure~\ref{fig:wordannotation} also shows that FoLiA has provisions for
encoding alternatives: an alternative sense is encoded alongside the actually
selected sense. This example illustrates alternatives for token annotation, yet a similar
construct is available for span annotation.


\subsection{Metadata}
\label{sec:metadata}

A FoLiA document always starts with a metadata specification. FoLiA introduces
a simple native $field \rightarrow value$ map for metadata, leaving it up to
the user to invent fields. However, more sophisticated metadata specifications
such as CMDI \cite{CMDI} can be used with FoLiA and are highly recommended.
These are independent of FoLiA and can be either embedded in the FoLiA
document, or be kept in an external file to which an explicit reference is
made. The motivation is to keep FoLiA specialised on actual representation
of annotations and leave metadata to more capable formats that are already
being used in the field.

%Although recommended, there is no obligation to specify any metadata at all
%except for 
A mandatory part of the metadata that can not be delegated to other metadata
formats yet is the declaration of annotation types and sets. This enables users
and parsers alike to know whether certain annotations are present in the
document, without resorting to a full scan. 
%It furthermore allows defaults for certain common attributes to be set so they
%need not be repeated throughout the document.

The common FoLiA attributes \texttt{annotator} and
\texttt{annotatortype} can arguably be considered metadata as well, but they are so closely
related to the annotations that their inclusion in FoLiA is justified in our view. In
fact, standardised facilities to easily make explicit the annotators of an
annotation are lacking in certain other formats, whilst they make a format more
suited for use in annotation tools.

%After the metadata specification a FoLiA document will start with the mandatory
%structural element \texttt{<text>} or \texttt{<speech>} within which all
%structure elements and annotation elements are contained.


\subsection{Higher Order Annotation}
\label{sec:higherorder}
\label{sec:features}

The paradigm of sets and classes allows for further granularity by introducing
\emph{features} and \emph{subsets}. A feature in FoLiA is a form of
higher-order annotation: it is essentially an annotation on an annotation.
Where token and span annotations take a set, features take a subset. This
subset is defined within the scope of a set, and thus in the same set
definition file. Like sets, subsets consist of classes.

An illustrative example of the usage of features and subsets can be made using
part-of-speech annotation, as shown in Figure~\ref{fig:features}. The
part-of-speech annotation element in this label set happens to be assigned a
class that encodes all features in one string, this is a property of the label
set and not of FoLiA, as FoLiA never prescribes classes. The feature notation
in FoLiA in turn allows you to explicitly encode each part-of-speech feature. 


\begin{figure}[tbh]
\begin{lstlisting}[language=xml]
    <pos class="N(soort,ev,basis,zijd,stan)" head="N">
        <feat subset="ntype" class="soort" />
        <feat subset="number" class="ev" />
        <feat subset="degree" class="basis" />
        <feat subset="gender" class="zijd" />
        <feat subset="case" class="stan" />
    </pos>
\end{lstlisting}
\caption{Example of a Part-of-Speech tag with features.\label{fig:features}}
\end{figure}

Figure~\ref{fig:features} additionally demonstrates that annotation types may \emph{predefine} certain subset
names (but never their classes): for part-of-speech annotation, the subset ``head''
is predefined, this then allows shorter notation using an XML attribute instead
of the \texttt{feat} element, but it is semantically identical within FoLiA.  Likewise,
Figure~ \ref{fig:textcontent} included an example of morphology, which predefines
the subset ``function''. There is no obligation on sets to actually use these
predefined subsets and any subset can be invented by the user, fully committing
to the idea of openness. % te veel detail?

Other forms of higher-order annotation which we shall not address further are:

\begin{itemize}
  \item \emph{Alignments} -- Links \cite{XLINK} between arbitrary annotations or structure
elements, also across FoLiA documents.  
  \item \emph{Descriptions} -- Extra human-readable descriptions can be associated with any annotation.
  \item \emph{Corrections} -- FoLiA also implements extensive support for storing and tracking
corrections or suggestions for correction.
\end{itemize}

\section{Comparison}
\label{sec:comparison}

In this section we will provide a more in-depth comparison between FoLiA and
other formats for linguistic annotation.

\subsection{The Linguistic Annotation Framework (LAF, GrAF, MAF, SynAF)}
\label{sec:LAF}

The Linguistic Annotation Framework is developed by the ISO/TC37/SC4
subcommittee of the International Organization for Standardisation (ISO). This
committee has developed ``principles and methods for creating, coding,
processing and managing language resources, such as written corpora, lexical
corpora, speech corpora, dictionary compiling and classification schemes''
\cite{LAF}. 

LAF can not be seen as a format as such, but rather as a comprehensive
meta-model or paradigm. The scope of the paradigm is wide, and the levels of
abstraction, generalisation, uniformity and consistency are high. Various user
formats or representations can in their own way make use of this data model.
This is in line with the observation that different constraints are posed on
formats by different uses.  There is an explicit aim at maximum flexibility. 

\citeasnoun{LAF} state that the LAF data model is essentially a feature
structure graph. An instantiation of this idea is found in the GrAF format
\cite{GRAF}, an actual XML-based format that can encode the full LAF data
model. This format consists in essence of nodes and edges. GrAF is intended to
act as the LAF pivot format for conversion between user-defined formats. 

The LAF paradigm is stand-off. It first of all lays a base segmentation
(tokenisation) by referencing to offsets in the original data document, a
separate file. Unlike FoLiA, multiple segmentations are possible. Annotations
(graph nodes) can then refer (using graph edges) to this base segmentation.
Annotation can also be drawn over other annotations, a more abstract and
generic version of what higher-order annotation in FoLiA is.

LAF-compatible representation of tokenisation and tagging is covered by the
Morpho-syntactic Annotation Framework (MAF) \cite{MAF}. The Syntactic
Annotation Framework (SynAF) focusses on the representation of syntactic
structures \cite{SYNAF}, and part of its focus is a definition of data
categories, which is outside the scope of FoLiA. SynAF is still a meta-model
and does not commit to a particular XML serialisation. However, SynAF is
inspired by Tiger XML \cite{TIGER}, the successor of which, Tiger2
\cite{TIGER2}, is a full SynAF-compatible serialisation format.

LAF and GrAF offer great flexibility but are of less direct practical use for
end-users seeking a hands-on format. With regard to this
\citeasnoun{TEICORPUSANNOT} remark: ``A tendency may be observed of increasing
abstractness and generality of proposed standards, esp., SynAF and LAF. This
leads to their greater formal elegance, at the cost of their actual
usefulness''. We concur and conclude that this is not a format end-users can
just pick up and use with readily available tools, despite the availability of
some libraries for GraF. Due to the level of abstraction, the learning curve is
fairly steep and leaves a lot to the user to define. Moreover, accessibility is
greatly hindered by the fact that the full and latest specification of LAF, MAF
and SynAF are not freely accessible, but only after purchase from ISO.  This is in sharp contrast with the emphasis on open access and open source in FoLiA. 

The question that arises in this context is to what extent FoLiA is compatible
with LAF, i.e. to what extent can FoLiA be seen as a format adhering to the LAF
data model and whether it can be converted to and from the pivot format proposed by LAF?

The fact that FoLiA in part commits to an inline annotation style is not an
obstacle for conversion to any fully stand-off representation desired by LAF;
more important is the fact that the notation employed by FoLiA is consistent
and unambiguous. The reverse conversion, from LAF to FoLiA, will be somewhat
more constrained. FoLiA will not be able to encode the full extent of LAF, most
notably it will not handle annotations on multiple mutually incompatible
tokenisations of the source text, unless one produces various and mutually
divergent FoLiA versions of the same document. The high generalisation and
flexibility in LAF may not always translate to FoLiA, as FoLiA is by design
more specific, less abstract, and facilities for annotation types have to be
explicitly present. Nevertheless, we think a large subset can still be covered
in a variety of applications, as many features are present in FoLiA and many
annotation types are supported. A large extent of the higher-order graph-based
annotations in LAF could also be represented in FoLiA through \emph{alignments}
and \emph{subsets/features}. Furthermore, FoLiA commits to a high level of
openness and, like LAF, takes standards such as ISOcat \cite{ISOCAT} into
consideration. %Possible future study and implementation of conversion to the
%LAF pivot format could test this thesis more empirically if there is demand for
%such a study. %dit is onnodig geblaat ;0)

%mre: note: heb halve zin toegevoegd over verschillende versies in FoLiA van zelfde doc met ander tok

\subsection{The Text Encoding Initiative (TEI P5)}

The Text Encoding Initiative has established itself as a de facto standard for
the encoding of a wide variety of text documents and is frequently used in the
Digital Humanities and beyond. The project represents a large community-based
effort which produces \emph{guidelines} for text encoding \cite{TEI}. It
defines a vast amount of tags for the encoding of all kinds of texts, in which
different views on a text can be taken for different uses and applications.
Like FoLiA, TEI is used for data storage, exchange, and processing.
Furthermore, TEI can be considered a direct ancestor leading to LAF and its
children: ``the current standards that have been or are being established by ISO
TC 37/SC4 committee ..., known together as the LAF (Linguistic Annotation Framework) family of standards, ... descend
in part from an early application of the TEI.'' \cite{TEICORPUSANNOT}. 

FoLiA's predecessor D-Coi states to be loosely based on a subset of TEI
\cite{DCOI}. FoLiA inherits this, which results in certain similarities in
structure annotation. Moreover, like FoLiA, TEI largely commits to an inline
annotation style, but also has support for stand-off notation where necessary.
When it comes to the encoding of textual structure, TEI clearly offers more
expressivity than FoLiA. 

TEI also provides facilities for linguistic annotation. Certain linguistic
properties such as lemmas can be represented as XML attributes.  Morphemes are
also well supported in TEI. A further set of generic analysis elements is
available, supporting spans too. Generic elements for \emph{feature structures}
are also available. These feature structures follow the same de jure standard
as the Morpho-syntactic Annotation Framework. \citeasnoun{TEILING} defines
feature structures as general-purpose data structures consisting of a named
feature and its value(s). Complex feature structures contain a group of
individual features allowing for a representation of various kinds of
information. FoLiA's token and span annotation elements, optionally enriched
with FoLiA's \emph{features}, are essentially (instantiations of) feature
structures; at a higher level of specificity and lower level of abstraction,
each annotation-type with proper semantics and following the same uniform
paradigm.

\citeasnoun{prze09} argues for TEI P5 as a suitable format for the encoding of
treebanks. He does observe that, at the time of the paper, the impact of the P5
version for linguistically annotated corpora is limited: ``Ideas useful for
linguistically annotated corpora are scattered over the 1350-odd pages of the
Guidelines, and usually there is more than one way of representing any given
annotation, so designing a coherent and constrained TEI-conformant schema for
linguistic corpora is a daunting task''.  It is unclear to what extent TEI has
been more enthusiastically adopted for linguistic annotation in recent years
in spite of this. Still, this is in sharp contrast to FoLiA's higher specificity.


\subsection{The D-SPIN Text Corpus Format (TCF 0.4)}

The D-SPIN text corpus format \cite{TCF} was developed in the context of the
Deutsche Sprachressourcen-Infrastruktur (D-SPIN) project, which is the
German contribution to the European Common Language Resources and Technology
Infrastructure (CLARIN) project. In comparison, FoLiA is used by several
projects in the Dutch version of CLARIN, i.e. CLARIN-NL  \cite{CLARINNL}.

The TCF format aims at a representation that supports interoperability between
different web services in a corpus processing pipeline. It is strongly linked to
the WebLicht software which offers a webservice infrastructure and employs
TCF as interchange format \cite{WEBLICHT}.

Of all the formats compared in this section, the TCF format is most similar to
FoLiA with regard to its intended use and audience. The levels of abstraction in
TCF and FoLiA are very similar, and both share a practical focus.

TCF is inspired by the Linguistic Annotation Framework \cite{LAF}, and unlike
FoLiA, the TCF format is primarily a stand-off format. The corpus text, its
tokens, and its annotations are all encoded in different sections.  This
implies that TCF differs from FoLiA in the dimension of \emph{locality} and
that \emph{parsability} of TCF may be somewhat more complicated than FoLiA as more
dereference operations are needed.

Like FoLiA, TCF assumes one basic tokenisation underlying the data. TCF also
employs globally unique identifiers, which are extensively used due to the
stand-off nature.

TCF differs from FoLiA in \emph{expressivity}, being less expressive than FoLiA
as FoLiA offers more annotation types. This makes FoLiA generally more verbose.
Annotation types in FoLiA are also more uniform than in TCF, as TCF does not
introduce as strict a paradigm as FoLiA does with its notion of sets, subsets,
classes and common attributes\footnote{Features such as support for encoding
who were the annotators do not seem to be available in TCF}.  All in all, FoLiA
offers a higher degree of formality, through set definitions and deep
validation.

TCF is developed with more regard for the compatibility with the Linguistic
Annotation Framework \cite{LAF} and sees more effort towards providing converters to this end
than FoLiA. For FoLiA this remains open for future work, but such conversions
from and to LAF standards should in principle be no less feasible than for TCF.

%TCF has some tools and libraries available, including validators. There is a Java software library as well as a Java pilot desktop application (TIEWER) for viewing and limited editing, the latter is not available for FoLiA yet. An online web application for viewing TCF is also available.

TCF has some tools and libraries available, including validators. There is a
Java software library as well as a Java pilot desktop application (TIEWER) for
viewing and limited editing. An online web application for viewing TCF is also
available.

%mre: Zal er wellicht binnenkort zijn in Nederlab maar heeft geen zin dat hier te zeggen

\subsection{Other formats}

We compared against two major formats or frameworks, and one more practical format which is
more on par with FoLiA. In this last section of our comparison, we will briefly
sketch some comparisons with other common formats in the field.

The XCES (XML Corpus Encoding Standard) format is an older format that is
derived from its non-XML predecessor CES \cite{XCES} and which aims to follow
the TEI guidelines. The format is fairly old, even in its second version, and
currently seems not well maintained, the last update of the xces.org site dates
from June 2008. It is criticized for its lack of documentation
\cite{TEICORPUSANNOT}.  \citeasnoun{prze09} anticipates this and adds that XCES
lacks specific recommendations for any linguistic levels, resorting instead to
general feature structure mechanisms.  FoLiA in contrast has explicit support
for various annotation types.

PAULA (Potsdamer Austauschformat Linguistischer Annotationen) is an XML format geared towards data exchange, used and developed within the SFB 632 project in Germany\footnote{\url{http://www.sfb632.uni-potsdam.de/}}. The format is fairly well documented and comes with DTD schemas for validation. It uses a strict stand-off style in which different layers of linguistic annotation are encoded in different files. A primary data text file forms the lowest level of resource representation \cite{PAULA}, and is always mandatory. A tokenisation file posits a segmentation, by verbose references, \emph{marks} in PAULA, to the
data file. Further marks or the more powerful and abstract \emph{structs} can
be used akin to span annotation in FoLiA.  Stand-off feature structures hold
the actual linguistic annotations. PAULA is more generic and less specific than
FoLiA. The stand-off nature makes it generally more verbose, less
human-readable and less easily parsable than FoLiA. It commits to a high level
of openness, but in this regard it lacks in formality as it does not seem to
provide facilities to define data categories or relate them to standards such
as ISOcat.  Unlike FoLiA, PAULA does not specify any real structure notation.

PAULA is inspired on early drafts of LAF and uses the same graph paradigm, but
is of more direct practical use. An open-source search and visualisation tool,
ANNIS2\footnote{\url{http://www.sfb632.uni-potsdam.de/d1/annis/}}, is available
with support for PAULA.


\section{Practice}
\label{sec:practice}

FoLiA emerged at the intersection of different projects, several still ongoing, within the Computational
Linguistics community in the Netherlands and Flanders. As part of the general overview of FoLiA in this paper, we in
this section present a short overview of projects and tools in which FoLiA is
used.
 
Especially notable is the SoNaR-500 corpus \cite{StevinSONAR2013}, the 540 million
word reference corpus for contemporary written Dutch, which is delivered in
FoLiA format. Other corpora in which FoLiA is used are DutchSemCor, a corpus
annotated with lexical semantic senses \cite{DUTCHSEMCOR};
Basilex\footnote{\url{http://www.basilex.nl}}, a corpus consisting of
Dutch texts young children would typically be exposed to; VU-DNC, a 2 million word diachronic corpus for Dutch offering both sentiment annotations and a gold standard for OCR post-correction.
Moreover, several projects within CLARIN-NL's TTNWW project are also employing FoLiA for
their representational needs. NWO project Political Mashup\footnote{\url{http://politicalmashup.nl}} embeds FoLiA in its own XML
format and has extended the format with elements for ``semanticising'' the
texts of the Dutch Acts of Parliament\footnote{E.g. Dutch parliamentary
proceedings 1930-2012, semanticised:
\url{https://easy.dans.knaw.nl/ui/datasets/id/easy-dataset:51848}}. In view of
the focus on Dutch in all these projects, it has to be stressed again that
FoLiA in fact is fully language independent. This is exemplified by the various
third-party projects at research group LT3\footnote{\url{http://lt3.hogent.be}}, Ghent
University, which have adopted the format for the projects AMiCA, PARIS and SubTLe \footnote{\url{http://lt3.hogent.be/en/projects/}}. In so far as these projects are collaborations between the various Flemish universities and research groups, FoLiA seems set to find strong footing over the full Dutch language area comprising both the Netherlands and Flanders.

One of the emphasized assets of FoLiA is the availability of tools to work with
the format, we believe this is a key aspect for projects considering adoption
of our format. Other factors are the fact it is actively supported, used, and
developed and the familiarity with its predecessor format D-Coi. There are
currently two complete FoLiA libraries available for programmers, a Python library
and a $C++$ library\footnote{Both libraries can be obtained from
\url{http://proycon.github.io/folia}}. Furthermore, a suite called
\emph{``FoLiA tools''}\footnote{\url{http://pypi.python.org/pypi/FoLiA-tools}}
is available; it contains several converters to other formats (including plain
text, CSV and HTML for visualisation), a validator, a tool for merging multiple
FoLiA documents and a command-line tool for limited querying/searching of FoLiA
documents. The visualisation stylesheet is popular and has already been adapted
by several third party projects such as NederLab, Basilex, and Political Mashup
to suit their specific needs.

Some further notable open-source tools\footnote{We list the tools in chronological order of their development and/or adaptation to FoLiA. The authors were not personally involved in the development of the last two listed.} that use FoLiA are:

\begin{itemize}
  \item Ucto\footnote{\url{http://ilk.uvt.nl/ucto}} - A multilingual rule-based
    tokeniser \cite{UCTO}. This tool is a good stepping stone for setting up an initial
    FoLiA document from scratch, it allows for  tokenising plain-text input.
  \item Frog\footnote{\url{http://ilk.uvt.nl/frog}} - An integration of various memory-based NLP modules developed for Dutch. Allows for plain-text Dutch input to be converted into richly
  annotated FoLiA documents. Uses FoLiA as its internal format.
 \item TICCL - Text-Induced Corpus Clean-up. A system geared at fully automatic lexical normalization and correction of typographical and Optical Character Recognition (OCR) misrecognition errors in possibly very large corpora \cite{Reynaert2010}.
  \item CLAM\footnote{\url{http://proycon.github.io/clam}} - CLAM allows the quick
   building of fully fledged RESTful webservices by wrapping around any
   command-line tool \cite{CLAM}. Though CLAM can be used with any format, it
   is very suitable for use with FoLiA as format for data exchange between
   webservices. This setup is found for instance in CLARIN-NL's and CLARIN Flander's joint TTNWW project. 
 \item Valkuil\footnote{\url{http://www.valkuil.net}} - A spelling corrector for Dutch.
 \item Fowlt\footnote{\url{http://www.fowlt.net}} - A spelling corrector for English.
 \item FoLiA-stats\footnote{to be released soon} - Frequency list generation and n-gram
   extraction on FoLiA documents, works on word forms, lemmas and POS-tags.
 \item BlackLab\footnote{\url{https://github.com/INL/BlackLab}} A corpus retrieval
   engine built on top of Apache Lucene. It allows fast, complex searches with
   accurate hit highlighting on large, tagged and annotated, bodies of text.
   Has support for both FoLiA as well as TEI. It will form the back-end for OpenSonaR.
 \item Brat\footnote{\url{http://brat.nlplab.org}} - Brat Rapid Annotation
   Tool. This web-based tool for structured text annotation is being extended at LT3, Ghent University, with support for importing, visualizing and exporting FoLiA information.
 \end{itemize}

At the moment, off-the-shelf annotation environments are not yet available for
FoLiA and remain much desired future work; promising work to this extent is
being performed in the context of the Brat
Rapid Annotation Tool. Corpus search environments that also include
FoLiA support are under development in the newly started projects
NederLab\footnote{\url{http://www.nederlab.nl/}} and OpenSoNaR\footnote{CLARIN-NL
Call 4 project which started October 2013 in which the Stevin SoNaR corpus is to
be made available online with various user interfaces for users with different
research interests}

From the usage of FoLiA in practice we can already draw some lessons. FoLiA and
XML are by nature very verbose formats, which results in sizable XML files.
This makes a strong demand on disk space as well as on memory if a document is
to be fully loaded in memory. The former problem can be remedied by
compression, which the FoLiA libraries support natively\footnote{GZip and BZip2
compression is supported, achieving very high compression factors}. The latter
problem can only be remedied by using streaming parsers, the principle of
locality ensures that such parsers can be suitably used on FoLiA, which can not
be said for most other formats.

Another cautionary lesson we can draw is that third party users are often quickly
inclined to either create ad-hoc extensions to the format or embed a subset of
FoLiA within different formats. Especially the former is problematic and
discouraged as FoLiA is a format high in specificity. Flexibility is offered in the form of the
set/class paradigm and corresponding set definition files, features in subsets offer even
more flexibility. If annotation categories need to be added, then this has to
be a more centralised development effort in FoLiA as a whole. This proves not
always sufficiently clear, and it is the task of the authors and developers to
make this clearer.

Embedding FoLiA in other formats, or even other formats in FoLiA, may be a
valid tactic on the strict condition that properly distinguishable XML
namespaces are used. However, in the former case, the resulting documents can
not be considered FoLiA documents as-such and can not be processed by the FoLiA
validator and other FoLiA tools; specialised parsers will be required.
Moreover, special care is in order not to lose the annotation declarations
required by FoLiA. 

Another issue is the fact that the limits of any XML-based format are quickly
reached in applications of real-time corpus querying. Real time iteration over
millions of XML documents is far too costly in terms of processing time. For
such applications, smart indexing and storage in some sort of database system
is indispensable. The aforementioned Nederlab and OpenSoNaR projects aim for
this. 

\section{Conclusion \& Future work}

There is plenty to choose from for anybody deciding on a format for
linguistic annotation. However, the abundance of choice does not necessarily make
things easier and it greatly depends on the particular goals and requirements of the
project.  

Our aim in this paper was twofold. Firstly, we have provided a descriptive
overview of the Format for Linguistic Annotation (FoLiA). Secondly, we have
conducted a comparative study with other formats.

FoLiA distinguishes itself from most other formats by choosing to represent
all annotations in a single XML file, following an inline annotation style where
possible, and embedding stand-off annotation layers locally. It furthermore
introduces a strict notion of sets and classes, and applies this consistently
across all annotation types. This proves to be a powerful paradigm.

Comparisons were drawn with the Linguistic Annotation Framework and its
children, the de jure standard for linguistic annotation. LAF offers a far
higher level of abstraction and genericness than FoLiA. Undoubtedly more
thought and debate has gone into establishing these theoretical foundations
than is the case in the development of FoLiA, which follows a completely
reversed bottom-up approach. FoLiA however aims at a far more practical level and is
more easily accessible than LAF, which makes it far more easy to adopt in
software tools. 

Similar conclusions can be drawn from the comparison with TEI. FoLiA offers
higher specificity for linguistic annotation\footnote{TEI though offers higher
specificity and expressivity for structure annotation} and is again at a more
practical level when off-the-shelf solutions are needed.

On a practical level, TCF and PAULA are most comparable with FoLiA. One thing
that sets FoLiA apart from all these formats is its commitment to locality, by
following a largely in-line and single-document setup. This eases
parsability and human readability. A second distinguishing characteristic of
FoLiA is its higher specificity; each annotation type is an explicit part of
the format and carries proper semantics. This implies a slightly lower level of
openness and genericness. The class/set and common attributes paradigm balances
this and ensures the necessary flexibility.

Future work on FoLiA focusses on one hand on increased expressivity of the
format, by adding specifications for more annotation types where current
facilities prove insufficient; and on the other hand on development of further
tools and libraries to work with the format. Current developments are
underway for better representation for linguistic annotation of speech, and for 
reducing the currently present bias towards written text. Further focus is on
finalising the more formal aspects of FoLiA; the set definitions and resulting
ability for deep validation. These are not yet embraced in current practice,
but are necessary for formal closure.

In line with the development of further tools is also the development of
converters to other formats. Especially increased connectivity with the
Linguistic Annotation Framework may be a notable point for future work.

\section{Acknowledgments}

Maarten van Gompel acknowledges CLARIN-NL for funding FoLiA support and
development within the scope of various projects, starting with the TTNWW
project. Martin Reynaert acknowledges support from NWO in projects Political
Mashup and Nederlab. FoLiA and TEI interoperability is being extended with
funding from CLARIN-NL in project @PhilosTEI (CLARIN-NL-12-006), while FoLiA
support in corpus search environments is further being developed in OpenSoNaR
(CLARIN-NL-12-013). We also acknowledge Ko van der Sloot as key developer for
FoLiA, contributing greatly to its flourishing, and Prof. Antal van den Bosch for his
continuing support for FoLiA. Last, we thank numerous partners in
the Dutch-Flemish NLP community for thinking with us about the format and for
enthusiastically adopting it.  


\bibliographystyle{clin} 
\bibliography{folia}  

\end{document}



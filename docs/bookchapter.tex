\chapter{A practical format and infrastructure for Linguistic Annotation}

\section{Introduction}

CLARIN's aim to deliver an infrastructure for researchers that work with
language data and tools can not be made possible without agreeing on standards
with regard to data formats. Standardisation is an important prerequisite for
good interoperability between the many language tools that emerged in the scope
of the CLARIN project, and for dissemination of the data set constructed in the
scope of CLARIN.

In the field, we often encounter an abundance of \emph{ad-hoc formats}. These are
data formats that are often characterised by one or more of the following traits:

\begin{itemize}
    \item They are only used once, often by one specific tool or for just one specific purpose.
    \item They are poorly formalised or not formalised at all, i.e. there is a lack of a formal schema and semantics.
    \item They are poorly documented.
    \item They are often rigid and hard to extend.
\end{itemize}

The use of such ad-hoc formats can be considered the opposite end of proper standardisation and is to be avoided. 

CLARIN adheres to the following principles when it comes standardisation:

\begin{itemize}
    \item Open standards are preferred over proprietary standards
    \item Formats and protocols should be:
    \begin{itemize}
        \item well-documented
        \item verifiable
        \item proven (being used in practice)
    \end{itemize}
    \item Text-based formats are (where possible) preferred over binary formats
\end{itemize}

With respect to linguistically annotated text, however, the Dutch and Flemish
NLP community lacked such a proper standard and ad-hoc formats were prevalent
in the field.  In the scope of CLARIN-NL project TTNWW, the NWO project
DutchSemCor, and the STEVIN project SoNaR, a new format for linguistic
annotation was developed as a solution to accommodate their representational needs, and so
FoLiA (Format for Linguistic Annotation) was created.

The aim of FoLiA is to provide a practical standard, following a generic
paradigm, for the linguistic annotation of primarily written text. To this end,
a wide variety of linguistic annotation types is supported. 

An extensive overview of the FoLiA format is given in \cite{FOLIAPAPER}, as
well as a comparison with competing standards and motivation for the creation
of FoLiA. Full documentation of the format can be found in \cite{FOLIADOC}.
Only a brief summary of key features will be repeated in Section~\ref{sec:overview}.

In the remainder of this current chapter, we intend to focus on the
\emph{practical} nature of the format instead, or rather, on the infrastructure
that is built around the format and the ways in which it had been put to use in
the scope of CLARIN and beyond.

\section{Overview}

FoLiA is an XML-based format and defines \emph{specific} XML elements for \textbf{structure
annotation} (e.g. paragraphs, sentences, word tokens, lists, figures..) and
\textbf{linguistic annotation} (e.g. part-of-speech, dependency relations,
syntax, named entites, etc..). FoLiA makes use of a combination of inline and
stand-off annotation, making proper use of the hierarchical nature of XML and
facilitating the job for parsers where possible. The format is fully language and
tagset independent as tagsets are defined seperately in \emph{FoLiA Set Definitions}
by users and never prescribed by FoLiA itself. Validation can proceed
on a shallow level, against a RelaxNG schema, as well as a deep level which
validates used tagsets against the set definition files.

The sets are at the core of the FoLiA paradigm, annotation elements take an
generic attribute named ``class''. These classes pertain to a set and are
defined by whatever set definition the user decides to use. The set definition
defines all allowed classes and allows for links with data category registries
for formal semantic closure.

Other generic attributes besides ``class'', are attributes to denote the
annotator of a particular annotation, the annotator type (human or machine),
the confidence level of the annotation, the time of the annotation, and more.

There are also has various types of \textbf{higher-order annotation}, such as
the ability to include alternative annotations, as well as extensive support
for corrections on annotations. There is also the ability to link other
modalities such as audio fragments of speech, to structural elements. So even
though FoLiA is a primarily a format to annotate text documents, speech
transcripts are supported as well. 

For metadata CLARIN has been committed to the CMDI standard \cite{CMDI}.
Although FoLiA has some rudimentary native support for metadata, we see no
sense in reinventing the wheel and FoLiA is ideally used in combination with an
external metadata format such as CMDI whenever extensive metadata is desired. A
reference to the metadata file can be made from within the FoLiA document.

\section{NLP tools}


















and attributes 


\section{}

\section{Conclusion}








































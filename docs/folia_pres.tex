\documentclass[compress]{beamer}
\usepackage{beamerthemeproyxetex}


\title{FoLiA: Format for Linguistic Annotation}
\author{Maarten van Gompel}
\date{01-02-2011}
\usepackage{graphicx}
\usepackage{listings}
\usepackage{color}
\lstset{% general command to set parameter(s)
basicstyle=\footnotesize,
keywordstyle=\color{black}\bfseries\underbar,
identifierstyle=\color{black}\bfseries\underbar,
stringstyle=\ttfamily,
}


\begin{document}

\begin{frame}
	\titlepage\smallraccoon\ilkuvt
\end{frame}

\section{Introduction}

\begin{frame}{Introduction}

    \begin{block}{Annotation formats in the field}
        \begin{itemize}
            \item Many ad-hoc annotation formats
            \item Many specialised annotation format
            \item Many conversions
            \item De facto-standard for various ILK projects: SoNaR/DCOI
        \end{itemize}
    \end{block}
    
    \begin{block}{Limits Frog columned format}
        \begin{itemize}
            \item Simplistic format, lacking expressiveness of XML
            \item Six/seven columns currently, covering full screen width
            \item Lacking even expressiveness to fully output Frog's data!
        \end{itemize}
    \end{block}

    \begin{block}{Limits of DCOI}
        \begin{itemize}
            \item Not expressive enough for many kinds of annotation (such as sense annotation, correction annotation)
            \item Can't encode annotators
            \item Can't encode alternatives
        \end{itemize}
    \end{block}

\end{frame}


\begin{frame}{Objectives}
    \begin{block}{Objectives}
        \textbf{Objectives: Expressability, Extensability, Uniformity}
        \begin{itemize}
            \item One generalised rich common XML format, supporting almost all we do at ILK
            \item Not committing to any particular tagset or language            
            \item Encoding many different annotation aspects similtaneously in a single file
            \item Support for sense annotation (DutchSemCor)
            \item Support for corrections (Vici project Tanja $+$ Ticcl)
            \item Support for NER (Project Steve)
            \item Support in Frog for reading and writing this format
            \item Inter-operability with ISO Data Category Registry
            \item Founded on the DCOI format (our de-facto standard)
            \begin{itemize}
                \item Backwards compatible with regard to identifiers
            \end{itemize}
            \item Open source, UTF-8 encoded,
        \end{itemize}        
    \end{block}
\end{frame}

\begin{frame}{Annotations}
    \begin{block}{Supported Annotations}

    It supports the following linguistic annotations:

    \begin{itemize}
        \item Part-of-Speech tags (with features)
        \item Lemmatisation
        \item Spelling corrections on both a tokenised as well as an untokenised level.
        \item Semantic sense annotation (to be used in DutchSemCor)
        \item Morphological Analysis
        \item Named Entities / Multi-word units
        \item Syntactic Parses
        \item Dependency Relations
        \item Chunking
        \item Semantic Role Labelling
        \end{itemize}

    \end{block}
\end{frame}

\begin{frame}{The Format: Overall skeleton}
    \begin{lstlisting}[language=xml]
    <?xml version="1.0" encoding="utf-8"?>
    <?xml-stylesheet type="text/xsl" href="http://ilk.uvt.nl/FoLiA/FoLiA.xsl"?>
    <FoLiA xmlns="http://ilk.uvt.nl/FoLiA"
      xmlns:xsi="http://www.w3.org/2001/XMLSchema-instance" xml:id="example">
      <!-- (Here IMDI or CMDI metadata can be inserted) -->
      <annotations>
          ...
      </annotations>    
      <text xml:id="example.text">
       <gap></gap>
       <body>
         ...
       </body>
       <gap></gap>
      </text>
    </FoLiA>  
    \end{lstlisting}
\end{frame}


\begin{frame}{The Format: Content elements}
    \begin{example}
     \begin{lstlisting}[language=xml]
         <p xml:id="TEST.p.1">
            <t>This is a test. It has two sentences.</t>
            <s xml:id="TEST.p.1.s.1">        
                <t>This is a test.</t>
                <w xml:id="TEST.p.1.s.1.w.1"><t>This</t></w>
                <w xml:id="TEST.p.1.s.1.w.2"><t>is</t></w>
                ..
            </s><s xml:id="TEST.p.1.s.2">        
                ..
          </p>                
     \end{lstlisting}
    \end{example}
    \begin{block}
    \begin{itemize}
        \item Paragraphs, Sentences, Words/Tokens
        \item Unique cummulative identifiers
        \item Text content element hold actual text.
    \end{itemize}
    \end{block}
\end{frame}



\begin{frame}{Paradigm}
    \begin{block}
        \begin{itemize}
            \item \textbf{Declarations}
            \item \textbf{Token Annotation}
            \item \textbf{Span Annotation}
        \end{itemize}
    
        \begin{itemize}
            \item \textbf{Set} - A particular tagset
            \item \textbf{Class} - A class in such a set
            \item \textbf{Annotator} - An open identifier for the user/system who provided the annotation
            \item \textbf{Annotator type} - ``auto'' or ''manual''
            \item \textbf{Confidence} - A confidence value between one and zero.
        \end{itemize}
            
    \end{block}

\end{frame}
